\documentclass{article}

\usepackage{amsmath,amssymb}
\usepackage{mathrsfs}
\usepackage{braket}

\title{Why Quantum Field Theory}
\author{Anthony Mezzacappa}
\date{August 20 and 25, 2020}

\begin{document}

\maketitle

\section{To treat physics that cannot be treated using non-relativistic quantum mechanics.}

\noindent \textbf{The fundamental shortcoming of non-relativistic quantum mechanics is its inability to treat systems where the number of particles changes.}

\qquad e.g., $ \gamma \gamma \leftrightarrow e^+ e^- $

\noindent Under relativistic conditions, we can \underline{expect this} to happen.

\noindent Consider a particle in a box of size $L$. The uncertainty in momentum is
\begin{equation*}
\Delta p \geq \dfrac{\hbar}{L}
\end{equation*}

\noindent For a relativistic particle
\begin{equation*}
E \simeq p c
\end{equation*}

\noindent Then
\begin{equation*}
\Delta p \geq \dfrac{\hbar}{L} \qquad \Rightarrow \qquad \Delta E \geq \dfrac{\hbar c}{L}
\end{equation*}

\noindent When
\begin{equation*}
\Delta E = 2 m c^2
\end{equation*}
where $m$ is the mass of the particle, the uncertainty in energy is above the threshold for particle-antiparticle production.

% Page 2

\noindent At what $L$ will this happen? When
\begin{equation*}
\Delta E = 2 m c^2 = \dfrac{\hbar c}{L}
\end{equation*}

\noindent Solving for $L$
\begin{align*}
\dfrac{\hbar c}{L} &= 2 m c ^2 \\
L &= \dfrac{\hbar c}{2 m c^2} = \dfrac{\hbar}{2mc}
\end{align*}

\newpage

\noindent The quantity
\begin{equation*}
\lambda_\mathrm{Compton} \equiv \dfrac{\hbar}{mc}
\end{equation*}
is the Compton wavelength.

\noindent $\Rightarrow$ When $ L \sim \lambda_\mathrm{Compton} $, we should expect to see a swarm of particle-antiparticle pairs surrounding the original particle (virtual particles, whose existence is limited by the uncertainty principle).

\noindent Note that
\begin{equation*}
\lambda_\mathrm{de~Broglie} \equiv \dfrac{\hbar}{p} > \lambda_\mathrm{Compton} = \dfrac{\hbar}{mc}
\end{equation*}
since $ p < m c $ ($ p = \gamma m_0 v = m v $).

\noindent $\lambda_\mathrm{de~Broglie}$ -- the distance at which the wavelike nature of the particle becomes apparent

% Page 3

\noindent $\lambda_\mathrm{Compton}$ -- the distance at which the concept of a single particle breaks down

\begin{center}
\begin{tabular}{|c|c|c|c|}
\hline 
$\lambda \ll \lambda_\mathrm{Compton}$ & $\lambda \sim \lambda_\mathrm{Compton}$ & $\lambda \sim \lambda_\mathrm{de~Broglie}$ & $\lambda \gg \lambda_\mathrm{de~Broglie}$ \\ 
\hline
& RQM a.k.a. QFT & NRQM & Classical \\
\hline
\end{tabular} 
\end{center}

\section{To treat physics that can be treated using quantum mechanics but more easily (naturally)}

\noindent \textbf{The treatment of systems of $N$ identical bosons or fermions ($ N \gg 1 $) in quantum mechanics is truly cumbersome (if not impossible for $N$ sufficiently large).}

\noindent We work with symmetrical (antisymmetrical) sums of products of single-particle wave functions. The symmetry (antisymmetry) is put in by hand.

\noindent There is a better way, known as the \underline{Number Representation}, which bundles $N$-particle systems in a much easier and more natural way. This representation is used in conjunction with quantum field operators, and symmetry (antisymmetry) is built into the theory!

\vspace{12pt}
\noindent \underline{\textbf{Other Reasons:}}

\section{Causality}

Consider the amplitude in quantum mechanics for a free particle to propagate from $ \vec{x} = \vec{x}_0 $ at $ t = 0 $ to $ \vec{x} $ at $t$:
\begin{align*}
& \braket{ \vec{x} | e^{ -i \hat{H} t } | \vec{x}_0 } \qquad \mathrm{position~eigenstate} \\
&= \braket{ \vec{x} | e^{ -i {[ \hat{p}^2 + m^2 ]}^\frac{1}{2} t } | \vec{x}_0 } \\
&= \int \mathrm{d}^3 \vec{p} ~ \braket{ \vec{x} | e^{ -i {[ \hat{p}^2 + m^2 ]}^\frac{1}{2} t } | \vec{p} } \braket{ \vec{p} | \vec{x}_0 }
\end{align*}

% Page 3a

\begin{equation*}
\int \mathrm{d}^3 \vec{p} ~ \ket{\vec{p}} \bra{\vec{p}} = 1 ?
\end{equation*}

\begin{align*}
& \braket{ \vec{x}^\prime | \left( \int \mathrm{d}^3 \vec{p} ~ \ket{\vec{p}} \bra{\vec{p}} \right) | \vec{x} } \\
&= \frac{1}{{(2 \pi)}^3} \int \mathrm{d}^3 \vec{p} ~ \braket{ \vec{x}^\prime | \vec{p} } \braket{ \vec{p} | \vec{x} } \\
&= \frac{1}{{(2 \pi)}^3} \int \mathrm{d}^3 \vec{p} ~ e^{ +i \vec{p} \cdot \vec{x}^\prime } e^{ -i \vec{p} \cdot \vec{x} } \\
&= \frac{1}{{(2 \pi)}^3} \int \mathrm{d}^3 \vec{p} ~ e^{ -i \vec{p} \cdot ( \vec{x} - \vec{x}^prime ) } \\
&= \delta^{(3)} ( \vec{x} - \vec{x}^\prime ) \\
&= \braket{ \vec{x}^\prime | \vec{x} }
\end{align*}

\begin{equation*}
\Rightarrow \quad \int \mathrm{d}^3 \vec{p} ~ \ket{\vec{p}} \bra{\vec{p}} = 1
\end{equation*}

% Page 4

\noindent For a free particle $[ \hat{H}, \hat{p} ] = 0 $.
\begin{align*}
\braket{ \vec{x} | e^{ -i \hat{H} t } | \vec{x}_0 } &= \int \mathrm{d}^3 \vec{p} ~ e^{ -i {[ p^2 + m^2 ]}^\frac{1}{2} t } \underbrace{\braket{ \vec{x} | \vec{p} }}_{ \dfrac{1}{{(2 \pi)}^\frac{3}{2}} e^{ -i \vec{p} \cdot \vec{x} } } \underbrace{\braket{ \vec{p} | \vec{x}_0 }}_{ \dfrac{1}{{(2 \pi)}^\frac{3}{2}} e^{ i \vec{p} \cdot \vec{x}_0 } } \\
&= \int \mathrm{d}^3 \vec{p} ~ e^{ -i {[ p^2 + m^2 ]}^\frac{1}{2} t } e^{ -i \vec{p} \cdot ( \vec{x} - \vec{x}_0 ) } \\
&= \int \mathrm{d}^3 \vec{p} ~ e^{ i \lbrace -{[ p^2 + m^2 ]}^\frac{1}{2} t - \vec{p} \cdot ( \vec{x} - \vec{x}_0 ) \rbrace }
\end{align*}

\noindent To make the math easy, let's assume we are in one spatial dimension, $x$, with $ x \gg t $ ( i.e., outside the light cone ) and $ x_0 = 0 $.
Then
\begin{align*}
\braket{ x | e^{ -i \hat{H} t } | 0 } = \int \dfrac{\mathrm{d} p}{2 \pi} ~ & \underbrace{e^{ -i {[ p^2 + m^2 ]}^\frac{1}{2} t  } e^{ -i p x }} \\
& \underbrace{e^{ i \left\lbrace -p - {[ p^2 + m^2 ]}^\frac{1}{2} {\left( \frac{x}{t} \right)}^{-1} \right\rbrace x }} \\
&\qquad \equiv e^{ i g(p) x } \quad \Rightarrow \quad g^\prime (p) = -1 - p {[ p^2 + m^2 ]}^{ -\frac{1}{2} } {\left( \frac{x}{t} \right)}^{-1}
\end{align*}
\begin{align*}
g^\prime (p) &= 0 ~ \mathrm{when} \\
-1 - p {[ p^2 + m^2 ]}^{ -\frac{1}{2} } {\left( \frac{x}{t} \right)}^{-1} &= 0 \\
-p {[ p^2 + m^2 ]}^{ -\frac{1}{2} } &= \dfrac{x}{t} \\
\dfrac{p^2}{ p^2 + m^2 } &= {\left( \frac{x}{t} \right)}^2 \\
p^2 &= {\left( \frac{x}{t} \right)}^2 \left( p^2 + m^2 \right) \\
\left[ 1 - {\left( \frac{x}{t} \right)}^2 \right] p^2 &= {\left( \frac{ x m }{t} \right)}^2 \\
p &= \pm \dfrac{ \frac{ x m }{t} }{ \sqrt{ 1 - {\left( \frac{x}{t} \right)}^2 } } = \pm \dfrac{ x m }{ \sqrt{ x^2 - t^2 } } \equiv p_\mathrm{s}
\end{align*}

% Page 5

\noindent The Method of Stationary Phase tells us that
\begin{align*}
I(x) &= \int_a^b e^{ i x g(p) } \mathrm{d}p \qquad \qquad x \gg 1, ~ g(p) \in \mathbb{R} ~ \mathrm{and} ~ \mathrm{smooth} \\
&\simeq e^{ i x g(p_\mathrm{s}) } \int_{-\infty}^{\infty} e^{ i x g'' (p_\mathrm{s}) p^2 } \mathrm{d}p \\
&= e^{ i x g(p_\mathrm{s}) } {\left( \dfrac{ 2 \pi i }{ x g'' (p_\mathrm{s}) } \right)}^\frac{1}{2}
\end{align*}
where
\begin{equation*}
    g' (p_\mathrm{s}) = 0
\end{equation*}

\noindent So, for us,
\begin{equation*}
    I (x) \simeq \frac{1}{2 \pi} e^{ i x g (p_\mathrm{s}) } {\left( \dfrac{ 2 \pi i }{ x g'' (p_\mathrm{s}) } \right)}^\frac{1}{2}
\end{equation*}
where
\begin{equation*}
    g (p_\mathrm{s}) = \left\lbrace \pm \dfrac{x m}{\sqrt{ x^2 - t^2 }} + {\left[ {\left( \pm \dfrac{x m}{\sqrt{ x^2 - t^2 }} \right)}^2 + m^2 \right]}^\frac{1}{2} {\left( \dfrac{x}{t} \right)}^{-1} \right\rbrace
\end{equation*}

% Page 6

\noindent The important point is
\begin{equation*}
    {\left| \braket{ x | e^{ -i \hat{H} t } | 0 } \right|}^2 = \dfrac{1}{ 2 \pi x | g'' (p_\mathrm{s}) | } \neq 0 ~ !
\end{equation*}
-- i.e., causality is violated.

\noindent We need to double check that $ | g'' (p_\mathrm{s} | $ is not $\infty$.
\begin{align*}
    g'' (p) &= \dfrac{\mathrm{d}}{\mathrm{d} p} \left\lbrace  -1 - p {[ p^2 + m^2 ]}^{-\frac{1}{2}} {\left( \dfrac{x}{t} \right)}^{-1} \right\rbrace \\
    &= - p {\left( \dfrac{x}{t} \right)}^{-1} \left( -\tfrac{1}{2} \right) 2p {[ p^2 + m^2 ]}^{-\frac{3}{2}} - {\left( \dfrac{x}{t} \right)}^{-1} {[ p^2 + m^2 ]}^{-\frac{1}{2}} \\
    &\stackrel{?}{=} \infty ~ \mathrm{for} ~ p = p_\mathrm{s} \\
    &= {\left. \dfrac{ p^2 - ( p^2 + m^2 ) }{ { ( p^2 + m^2 ) }^\frac{3}{2} } \right|}_{ p = p_\mathrm{s} } {\left( \dfrac{x}{t} \right)}^{-1} \\
    &= {\left. \dfrac{ - m^2 }{ { ( p^2 + m^2 ) }^\frac{3}{2} } \right|}_{ p = p_\mathrm{s} } {\left( \dfrac{x}{t} \right)}^{-1}
\end{align*}
\begin{align*}
    p_\mathrm{s}^2 + m^2 &= \dfrac{ x^2 m^2 }{ x^2 - t^2 } + m^2 \qquad ( x\gg t ) \\
    & \simeq 2 m^2
\end{align*}
\begin{align*}
    g'' (p_\mathrm{s}) &\simeq -\frac{1}{ {(2)}^\frac{3}{2} } \frac{1}{m} \frac{t}{x} \\
    {|g'' (p_\mathrm{s}) |}^2 &\simeq \frac{ \sqrt{2} m }{ \pi t } \longrightarrow 0 ~ \mathrm{as} ~ t \longrightarrow \infty
\end{align*}
No.

\section{Locality (related to causality)}

\noindent Instantaneous action at a distance is incompatible with relativity.

\noindent Classically, we've solved this problem for Coulomb's Law and Newton's Law of Gravitation by introducing fields, whose solution is governed by Maxwell's Equations and Einstein's Equations, respectively.

% Page 7

\noindent Why wouldn't all interactions in Nature (gravitational, electromagnetic, weak, and strong) be describable through fields?

\noindent When we consider quantized fields, we will see that interactions are mediated by particles of a particular spacetime point -- i.e., they are \underline{local}.

\end{document}
