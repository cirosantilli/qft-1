\documentclass{article}

\usepackage{amsmath,amssymb}
\usepackage{mathrsfs}
\usepackage{tensor}
\usepackage{fullpage}

\usepackage{textpos}
\usepackage{eso-pic}
\usepackage{tikz}
\usetikzlibrary{tikzmark}

\title{Minkowski Spacetime, Its Metric, and Tensor Operations}
\author{Anthony Mezzacappa}
\date{September 10, 2020}
% Typeset by Wileam Phan

\begin{document}
\setlength{\parskip}{1em}
\maketitle

\begin{align*}
    \tikzmarknode{ds2}{\mathrm{d} s^2} &= -c^2 ~ \mathrm{d}t^2 + \mathrm{d}x^2 + \mathrm{d}y^2 + \mathrm{d}z^2 \qquad \text{(assumes Cartesian coordinates)} \\
    \tikzmarknode{dt2}{\,} &= -c^2 ~ \mathrm{d}\tau^2 \\
    c^2 ~ \mathrm{d}\tau^2 &= c^2 ~ \mathrm{d}t^2 - \mathrm{d}x^2 - \mathrm{d}y^2 - \mathrm{d}z^2 \underset{c=1}{\longrightarrow} \mathrm{d}t^2 - \mathrm{d}x^2 - \mathrm{d}y^2 - \mathrm{d}z^2 \\
    \mathrm{d}\tau^2 &= - \tikzmarknode{eta}{\tensor{\eta}{_\mu_\nu}} ~ \mathrm{d} \tensor{x}{^\mu} ~ \mathrm{d} \tensor{x}{^\nu} \equiv \tensor{\bar{\eta}}{_\mu_\nu} ~ \mathrm{d} \tensor{x}{^\mu} ~ \mathrm{d} \tensor{x}{^\nu}
\end{align*}
{%
\begin{textblock*}{1in}(0.0in,-1.15in)%
\begin{minipage}[h!]{1in}
    \tikzmarknode{ds2label}{proper length}
\end{minipage}%
\end{textblock*}%
}%
{%
\begin{textblock*}{1in}(0.0in,-0.9in)%
\begin{minipage}[h!]{1in}
    \tikzmarknode{dt2label}{proper time}
\end{minipage}%
\end{textblock*}%
}
{%
\begin{textblock*}{3in}(1.55in,-0.2in)%
\begin{minipage}[h!]{3in}
    \tikzmarknode{metric}{M}inkowski metric = %
    \begin{pmatrix}
    -1 & 0 & 0 & 0 \\
    0  & 1 & 0 & 0 \\
    0  & 0 & 1 & 0 \\
    0  & 0 & 0 & 1
    \end{pmatrix}
\end{minipage}%
\end{textblock*}%
}
\begin{tikzpicture}[overlay, remember picture]
    \draw[overlay,->] (ds2label.east) -- (ds2.west);
    \draw[overlay,->] (dt2label.east) -- (dt2.west);
    \draw[overlay,->] (metric.north) -- (eta.south);
\end{tikzpicture}

\vspace{-12pt} \begin{align*}
    \tensor{\eta}{_0_0} &= -1 \\
    \tensor{\eta}{_i_i} &= 1 \\
    \tensor{\eta}{_\mu_\nu} &= 0 \qquad \mu \neq \nu \\
\end{align*}

\vspace{-48pt} \begin{align*}
    \mathrm{d}\tensor{x}{^0} &= \mathrm{d}t \\
    \mathrm{d}\tensor{x}{^1} &= \mathrm{d}x \\
    \mathrm{d}\tensor{x}{^2} &= \mathrm{d}y \\
    \mathrm{d}\tensor{x}{^3} &= \mathrm{d}z
\end{align*}

\noindent All tensors are built from contravariant vectors (vectors) and covariant vectors (covectors or 1-forms).

\section*{Contravariant Vector}
$\vec{V} : T_P M \rightarrow \mathbb{R}$
\begin{equation*}
    \vec{V} = \tikzmarknode{cvvmu}{\tensor{V}{^\mu}} \tensor{\hat{\mathbf{e}}}{_\mu} \qquad \tikzmarknode{mulabel}{\text{index}} \text{ \underline{ on top} (\underline{contra}variant)} 
\end{equation*}
\begin{tikzpicture}[overlay, remember picture]
    \draw[overlay,->] (mulabel.north) -- +(0,0.2) -- +(-1.5,0.2) -- (cvvmu.north east);
\end{tikzpicture}

% Page 2

\noindent N.B. The Einstein index convention is at play here:
\begin{equation*}
    \vec{V} = \underbrace{ \tensor{V}{^\mu} \tensor{\hat{\mathbf{e}}}{_\mu} } = \tensor{V}{^0} \tensor{\hat{\mathbf{e}}}{_0} + \tensor{V}{^1} \tensor{\hat{\mathbf{e}}}{_1} + \tensor{V}{^2} \tensor{\hat{\mathbf{e}}}{_2} + \tensor{V}{^3} \tensor{\hat{\mathbf{e}}}{_3}
\end{equation*}
\hspace{2.2in} sum over repeated indices

\section*{Covariant Vector}
$\tilde{V} : T_P M^* \rightarrow \mathbb{R}$
\begin{equation*}
    \tilde{V} = \tikzmarknode{1fmu}{\tensor{V}{_\mu}} \tikzmarknode{dxmu}{\tensor{\tilde{\mathrm{dx}}}{^\mu}}
\end{equation*}
{%
\begin{textblock*}{1in}(2.4in,0.0in)%
\begin{minipage}[h!]{1in}
    \tikzmarknode{mulabel2}{index down}
\end{minipage}%
\end{textblock*}%
}%
{%
\begin{textblock*}{3in}(3.5in,0.0in)%
\begin{minipage}[h!]{3in}
    \tikzmarknode{dxmulabel}{1}-form \underline{dual} to $\tensor{\hat{\mathbf{e}}}{_\mu}$ -- i.e., $\tensor{\tilde{\mathrm{dx}}}{^\mu} ( \tensor{\hat{\mathbf{e}}}{_\nu} ) = \tensor{\delta}{^\mu_\nu} $
\end{minipage}%
\end{textblock*}%
}
\begin{tikzpicture}[overlay, remember picture]
    \draw[overlay,->] (mulabel2.east) -- (1fmu.south);
    \draw[overlay,->] (dxmulabel.east) -- (dxmu.south);
\end{tikzpicture}

\vspace{-24pt} \begin{equation*}
    \tilde{V} (\vec{V}) = \tilde{V} ( \tensor{V}{^\alpha} \tensor{\hat{\mathbf{e}}}{_\alpha} ) = \tensor{V}{_\beta} \tensor{V}{^\alpha} \tensor{\tilde{\mathrm{dx}}}{^\beta} ( \tensor{\hat{\mathbf{e}}}{_\alpha} )
\end{equation*}

\noindent General Tensors are created via the \underline{Tensor Product}. Let's look at the important case of the \underline{Metric Tensor}

\end{document}