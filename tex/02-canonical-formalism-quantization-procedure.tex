\documentclass{article}

\usepackage{amsmath,amssymb}
\usepackage{mathrsfs}
\usepackage{braket}

\title{A Review of the Canonical Formalism and Quantization Procedure for Particles}
\author{Anthony Mezzacappa}
\date{August 27, 2020}

\begin{document}
\setlength{\parskip}{1em}
\maketitle

\noindent Consider the one-dimensional motion of a single particle in a conservative force field. Let $q$ be the generalized coordinate of the particle. Then
\begin{equation*}
    \dot{q} \equiv \dfrac{ \mathrm{d}q }{ \mathrm{d} t}
\end{equation*}
is the generalized velocity of the particle. Let $ L (q, \dot{q}) $ be the Lagrangian.

\vspace{12pt}
\noindent \underline{Hamilton's Principle}

\noindent The physical path $ q (t) $ that the particle takes in going from $ q (t_1) \equiv q_1 $ to $ q (t_2) \equiv q_2 $ is that along which the \underline{Action}, $S$, is stationary.

\noindent The action is defined by
\begin{equation*}
    S = \int_{t_1}^{t_2} L (q, \dot{q}) ~\mathrm{d}t
\end{equation*}

\noindent Along the physical path, small \underline{variations} in the path
\begin{equation*}
    q (t) \longrightarrow q (t) + \delta q (t)
\end{equation*}
leading to a variation of the action, $ \delta S $, leave the action unchanged to first order in the variation, $ \delta q (t) $, -- i.e.,
\begin{equation*}
    \delta S = 0
\end{equation*}

% Page 2

\noindent Let's compute $ \delta S $:

\begin{align*}
    \delta S &= \delta \int_{t_1}^{t_2} L (q, \dot{q}) ~\mathrm{d}t \\
    &= \int_{t_1}^{t_2} \left[ \dfrac{ \partial L }{ \partial q } \delta q + \dfrac{ \partial L }{ \partial \dot{q} } \delta ( \dot{q} ) \right] ~\mathrm{d}t \\
    &= \int_{t_1}^{t_2} \left[ \dfrac{ \partial L }{ \partial q } \delta q + \dfrac{ \partial L }{ \partial \dot{q} } \dfrac{ \mathrm{d} }{ \mathrm{d} t } ( \delta q ) \right] ~\mathrm{d}t \\
    &= \int_{t_1}^{t_2} \left[ \dfrac{ \partial L }{ \partial q } \delta q + \dfrac{ \mathrm{d} }{ \mathrm{d} t } \left( \dfrac{ \partial L }{ \partial \dot{q} } \delta q  ) \right) - \dfrac{ \mathrm{d} }{ \mathrm{d} t } \left( \dfrac{ \partial L }{ \partial \dot{q} } \right) \delta q \right] ~\mathrm{d}t \\
    &= \int_{t_1}^{t_2} \left[ \dfrac{ \partial L }{ \partial q } - \dfrac{ \mathrm{d} }{ \mathrm{d} t } \left( \dfrac{ \partial L }{ \partial \dot{q} } \right) \right] \delta q ~\mathrm{d}t
\end{align*}

\noindent The last equality arises because
\begin{equation*}
    \int_{t_1}^{t_2} \dfrac{ \mathrm{d} }{ \mathrm{d} t } \left( \dfrac{ \partial L }{ \partial \dot{q} } \delta q \right) ~\mathrm{d}t = \left. \dfrac{ \partial L }{ \partial \dot{q} } \delta q \right|_{t_2} - \left. \dfrac{ \partial L }{ \partial \dot{q} } \delta q \right|_{t_1} = 0
\end{equation*}
since $ \delta q (t_1) $ and $ \delta q (t_2) $ are by definition 0 given the definite starting point, $ q (t_1) \equiv q_1 $, and ending point, $ q (t_2) \equiv q_2 $.

% Page 2a

\noindent\rule{\textwidth}{.5pt}

\noindent Given a functional
\begin{equation*}
    F = F \left( x, y(x), y'(x) \right)
\end{equation*}
consider the variation
\begin{equation*}
    y(x) \longrightarrow y(x) + \epsilon \eta (x)
\end{equation*}
where $ \epsilon \ll 1 $.
\begin{equation*}
    \delta y(x) \equiv \epsilon \eta (x)
\end{equation*}

\noindent Then, at a \underline{fixed} x,
\begin{equation*}
    \delta F = F \left( x, y(x) + \epsilon \eta (x), y'(x) + \epsilon \eta' (x) \right) - F \left( x, y(x), y'(x) \right)
\end{equation*}

\noindent Expand in $\underline{\underline{\epsilon}}$:

\begin{equation*}
    \delta F = \dfrac{ \partial F }{ \partial y } \epsilon \eta + \dfrac{ \partial F }{ \partial y' } \epsilon \eta'
\end{equation*}

\noindent For
\begin{align*}
    F = y &\Longrightarrow \delta y = \epsilon \eta \\
    F = y' &\Longrightarrow \delta y' = \epsilon \eta' = {\left( \delta y \right)}'
\end{align*}
$\delta$ and $\tfrac{\mathrm{d}}{ \mathrm{d} x }$ commute!

\noindent\rule{\textwidth}{.5pt}

% Page 3

\begin{equation*}
    \begin{split}
        \int_{t_1}^{t_2} \left[ \quad \right] \delta q ~\mathrm{d}t &= {\left[ \quad \right]}_{t_i} \delta q (t_i) \Delta t + {\left[ \quad \right]}_{t_{i+1}} \delta q (t_{i+1}) \Delta t \\
        &\quad + \dots = 0 \quad \Longrightarrow \quad \mathrm{all}~ {\left[ \quad \right]}_{t_i} = 0
    \end{split}
\end{equation*}

\noindent Since the variation $ \delta q (t) $ of the function $q(t)$ is arbitrary, the physical path (for which $\delta S = 0$) is given by the solution of the Euler-Lagrange EOM
\begin{equation*}
    \dfrac{ \partial L }{ \partial q } - \dfrac{\mathrm{d}}{ \mathrm{d} t } \left( \dfrac{ \partial L }{ \partial \dot{q} } \right) = 0
\end{equation*}

\noindent The momentum conjugate to $q$ is defined by
\begin{equation*}
    p \equiv \dfrac{ \partial L }{ \partial \dot{q} }
\end{equation*}

\noindent The Hamiltonian, which is a function of $(q, p)$ rather than $(q, \dot{q})$, is defined by
\begin{equation*}
    H \equiv p \dot{q} - L
\end{equation*}

\noindent The EOM are (using the \underline{Poisson brackets})
\begin{align*}
    \dot{q} &= - \left\lbrace H, q \right\rbrace = - \left( \dfrac{ \partial H }{ \partial q } \dfrac{ \partial q }{ \partial p } - \dfrac{ \partial H }{ \partial p } \dfrac{ \partial q }{ \partial q } \right) = \dfrac{ \partial H }{ \partial p } \\
    \dot{p} &= - \left\lbrace H, p \right\rbrace = - \left( \dfrac{ \partial H }{ \partial q } \dfrac{ \partial p }{ \partial p } - \dfrac{ \partial H }{ \partial p } \dfrac{ \partial p }{ \partial q } \right) = - \dfrac{ \partial H }{ \partial q }
\end{align*}

\noindent To quantize the system, $q$ and $p$ become Hermitian operators
\begin{align*}
    q &\longrightarrow \hat{q} \quad \mathrm{whose} ~ \mathrm{action} ~ \mathrm{on} ~ \psi (q, t) ~ \mathrm{is} ~ \mathrm{multiplication} ~ \mathrm{by} ~ q \\
    p &\longrightarrow \hat{p} \equiv -i \frac{\partial}{ \partial q }
\end{align*}
and the Hamiltonian becomes a Hermitian operator. The Schr\"{o}dinger Equation is
\begin{equation*}
    \hat{H} \psi (q, t) = i \dfrac{ \partial \psi (q, t) }{ \partial t }
\end{equation*}
% Note: I decided not to break the paragraph, so I added the continuation part from Page 4 here, before moving on to page 3a. -WYP

\noindent\rule{\textwidth}{.5pt}

% Page 3a

\noindent Considering the Hamiltonian as a function of $(q, p)$, not explicitly $t$, we have
\begin{equation}
    \mathrm{d} H (p, q) = \dfrac{ \partial H }{ \partial p } \mathrm{d} p + \dfrac{ \partial H }{ \partial q } \mathrm{d} q \label{l2eq1} % Equation 1
\end{equation}

\noindent Considering the definition of the Hamiltonian
\begin{equation}
    H (p, q) \equiv p \dot{q} - L (q, \dot{q}) % Equation 2
\end{equation}
\begin{equation}
    \mathrm{d} H = p \,\mathrm{d} \dot{q} + \dot{q} \,\mathrm{d} p - \dfrac{ \partial L }{ \partial q } \mathrm{d} q - \dfrac{ \partial L }{ \partial \dot{q} } \mathrm{d} \dot{q} \label{l2eq3} % Equation 3
\end{equation}
where we have assumed that $L$ is not an explicit function of time (this would break Lorentz invariance of the Lagrangian, which is a scalar quantity, when we build our relativistic QFT).

\noindent From the Euler-Lagrange EOM
\begin{equation}
    \dfrac{ \partial L }{ \partial q } - \dfrac{\mathrm{d}}{ \mathrm{d} t } \left( \dfrac{ \partial L }{ \partial \dot{q} } \right) = 0 % Equation 4
\end{equation}

\noindent But
\begin{equation}
    p \equiv \dfrac{ \partial L }{ \partial \dot{q} } \label{l2eq5} % Equation 5
\end{equation}

\noindent The Euler-Lagrange EOM then tell us
\begin{equation}
    \dfrac{\mathrm{d}}{ \mathrm{d} t } (p) = \dot{p} = \dfrac{ \partial L }{ \partial q } \label{l2eq6} % Equation 6
\end{equation}

% Page 3b

\noindent Then, insertion of \eqref{l2eq5} and \eqref{l2eq6} in \eqref{l2eq3} gives
\begin{align*}
    \mathrm{d} H \equiv &= \left( p - \dfrac{ \partial L }{ \partial \dot{q} } \right) \mathrm{d} \dot{q} + \dot{q} \,\mathrm{d} p - \dfrac{ \partial L }{ \partial q } \mathrm{d} q \qquad \longleftarrow ~ \mathrm{just} ~ \mathrm{a} ~ \mathrm{rewrite} ~ \mathrm{of} ~ \eqref{l2eq3} \\
    &= \dot{q} \mathrm{d} p - \dot{p} \mathrm{d} q
\end{align*}

\noindent But, from \eqref{l2eq1}, equating coefficients
\begin{align*}
    \dot{q} &= ~ \dfrac{ \partial H }{ \partial p } \\
    \dot{p} &= - \dfrac{ \partial H }{ \partial q }
\end{align*}

\noindent These are Hamilton's EOM, which can be expressed in terms of the so-called Poisson Brackets as
\begin{align*}
    \dot{q} &= - \left\lbrace H, q \right\rbrace = - \left( \dfrac{ \partial H }{ \partial q } \dfrac{ \partial q }{ \partial p } - \dfrac{ \partial H }{ \partial p } \dfrac{ \partial q }{ \partial q } \right) = \dfrac{ \partial H }{ \partial p } \\
    \dot{p} &= - \left\lbrace H, p \right\rbrace = - \left( \dfrac{ \partial H }{ \partial q } \dfrac{ \partial p }{ \partial p } - \dfrac{ \partial H }{ \partial p } \dfrac{ \partial p }{ \partial q } \right) = \dfrac{ \partial H }{ \partial q }
\end{align*}

\noindent\rule{\textwidth}{.5pt}

% Page 4 (after the broken paragraph)

\noindent In the Heisenberg Representation, it's the operators that depend on time, not the states.
\begin{equation*}
    \psi_\mathrm{S} (q, t) = e^{ - i \hat{H} t } \psi_\mathrm{S} (q, 0) \equiv e^{ - i \hat{H} t } \psi_\mathrm{H}
\end{equation*}

\noindent The time-independent operators in the Schr\"{o}dinger picture are replaced in the Heisenberg picture by
\begin{equation*}
    \hat{O}_\mathrm{H} (t) = e^{ i \hat{H} t } \, \hat{O}_\mathrm{S} \, e^{ - i \hat{H} t }
\end{equation*}
and the time development of the operators is given by
\begin{equation*}
    \dfrac{ \mathrm{d} \hat{O}_\mathrm{H} }{ \mathrm{d} t } = i [ \hat{H}, \hat{O}_\mathrm{H} ]
\end{equation*}

\noindent\rule{\textwidth}{.5pt}

% Page 4a

\noindent Consider the time derivative of the expectation value of the operator, $\hat{O}_\mathrm{S}$, in the Schr\"{o}dinger picture in the state $\ket{ \psi_\mathrm{S} }$:
\begin{equation} \label{l2eq7}
    \dfrac{\mathrm{d}}{ \mathrm{d} t } \braket{ \psi_\mathrm{S} | \hat{O}_\mathrm{S} | \psi_\mathrm{S} }
\end{equation} % originally Equation 1
Here the state $\ket{ \psi_\mathrm{S} }$ is time-dependent but the operator $\hat{O}_\mathrm{S}$ is not.

\noindent We can re-express the matrix element in terms of $\ket{ \psi_\mathrm{H} }$ and $\hat{O}_\mathrm{H}$ as
\begin{equation} \label{l2eq8}
    \dfrac{\mathrm{d}}{ \mathrm{d} t } \braket{ \psi_\mathrm{H} | e^{ i \hat{H} t } \hat{O}_\mathrm{S} e^{ -i \hat{H} t } | \psi_\mathrm{H} }
\end{equation} % originally Equation 2

\noindent Now \underline{all} of the time dependence sits in the exponentials $e^{ \pm i \hat{H} t }$.

\noindent Taking the derivative
\begin{align}
    & \braket{ \psi_\mathrm{H} | \left( i \hat{H} e^{ i \hat{H} t } \hat{O}_\mathrm{S} e^{ -i \hat{H} t } - i e^{ i \hat{H} t } \hat{O}_\mathrm{S} \hat{H} e^{ -i \hat{H} t } \right) | \psi_\mathrm{H} } \nonumber\\
    &= \braket{ \psi_\mathrm{H} | i \left( \hat{H} e^{ i \hat{H} t } \hat{O}_\mathrm{S} e^{ -i \hat{H} t } - e^{ i \hat{H} t } \hat{O}_\mathrm{S} e^{ -i \hat{H} t } \hat{H} \right) | \psi_\mathrm{H} } \nonumber\\
    &= \braket{ \psi_\mathrm{H} | i \left( \hat{H} \hat{O}_\mathrm{H} - \hat{O}_\mathrm{H} \hat{H} \right) | \psi_\mathrm{H} } \nonumber\\
    &= \braket{ \psi_\mathrm{H} | i [ \hat{H}, \hat{O}_\mathrm{H} ] | \psi_\mathrm{H} } \label{l2eq9}
\end{align} % originally Equation 3

% Page 4b

Then, from \eqref{l2eq8} and \eqref{l2eq9}
\begin{align*}
     & \dfrac{\mathrm{d}}{ \mathrm{d} t } \braket{ \psi_\mathrm{H} | \hat{O}_\mathrm{H} | \psi_\mathrm{H} } \\
     &= \braket{ \psi_\mathrm{H} | \dfrac{\mathrm{d}}{ \mathrm{d} t } \hat{O}_\mathrm{H} | \psi_\mathrm{H} } \\
     &= \braket{ \psi_\mathrm{H} | i [ \hat{H}, \hat{O}_\mathrm{H} ] | \psi_\mathrm{H} }
\end{align*}

\noindent We arrive at the operator relation
\begin{equation*}
    \dfrac{ \mathrm{d} \hat{O}_\mathrm{H} }{ \mathrm{d} t } = i [ \hat{H}, \hat{O}_\mathrm{H} ]
\end{equation*}
which is the Heisenberg EOM for the time-dependent operator $\hat{O}_\mathrm{H}$.

\noindent\rule{\textwidth}{.5pt}

\noindent Let's look at $ \hat{O}_\mathrm{H} = \hat{p} (t)$. Using an arbitrary function, $ f (p, q) $, as a placeholder,
\begin{align*}
    \dfrac{ \mathrm{d} \hat{p} (t) }{ \mathrm{d} t } f &= i [ \hat{H}, \hat{p} ] f \\
    &= i \left[ \hat{H} \left( -i \dfrac{\partial}{ \partial q } \right) - \left( -i \dfrac{\partial}{ \partial q } \right) \hat{H} \right] f \\
    &= \hat{H} \left( \dfrac{ \partial f }{ \partial q } \right) - \dfrac{\partial}{ \partial q } \left( \hat{H} f \right) \\
    &= \hat{H} \left( \dfrac{ \partial f }{ \partial q } \right) - \dfrac{ \partial \hat{H} }{ \partial q } \left( f \right) - \hat{H} \dfrac{ \partial f }{ \partial q } \\
    &= - \dfrac{ \partial \hat{H} }{ \partial q } \left( f \right)
\end{align*}

% Page 5

\noindent Then
\begin{equation*}
    \dfrac{ \mathrm{d} \hat{p} (t) }{ \mathrm{d} t } = - \dfrac{ \partial \hat{H} }{ \partial q }
\end{equation*}

\noindent Now let's look at $ \hat{O}_\mathrm{H} = \hat{q} (t)$.
\begin{align*}
    \dfrac{ \mathrm{d} \hat{q} (t) }{ \mathrm{d} t } f &= i [ \hat{H}, \hat{q} ] f \\
    &= i \left[ \hat{H} \left( i \dfrac{\partial}{ \partial p } \right) - \left( i \dfrac{\partial}{ \partial p } \right) \hat{H} \right] f \\
    &= - \hat{H} \left( \dfrac{ \partial f }{ \partial p } \right) + \dfrac{\partial}{ \partial p } \left( \hat{H} f \right) \\
    &= - \hat{H} \left( \dfrac{ \partial f }{ \partial p } \right) + \dfrac{ \partial \hat{H} }{ \partial p } \left( f \right) + \hat{H} \dfrac{ \partial f }{ \partial p } \\
    &= \dfrac{ \partial \hat{H} }{ \partial p } \left( f \right)
\end{align*}

\noindent Then
\begin{equation*}
    \dfrac{ \mathrm{d} \hat{q} (t) }{ \mathrm{d} t } = \dfrac{ \partial \hat{H} }{ \partial p }
\end{equation*}

\noindent But these are just the \underline{classical equations of motion}.

\noindent $\Longrightarrow$ \quad In the Heisenberg representation, the relevant operators evolve according to the classical EOM.

\noindent This is very important in QFT and motivates the approach we will take to build our theories:
\begin{enumerate}
    \item CONSTRUCT LAGRANGIANS FOR CLASSICAL FIELDS
    \item DERIVE THE EOM FOR THOSE FIELDS
    \item FIND THE CLASSICAL SOLUTIONS TO THESE EOM
    \item QUANTIZE THESE CLASSICAL FIELD SOLUTIONS
    \begin{itemize}
        \item[-] ELEVATE THEM TO OPERATOR STATUS
        \begin{itemize}
            \item[-] INTRODUCE CREATION AND ANNIHILATION OPERATORS
            \begin{itemize}
                \item[-] MAKE THE CONNECTION TO PARTICLES
            \end{itemize}
        \end{itemize}
    \end{itemize}
\end{enumerate}

\end{document}
