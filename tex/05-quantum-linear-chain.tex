\documentclass{article}

\usepackage{amsmath,amssymb}
\usepackage{mathrsfs}
\usepackage{braket}
\usepackage{fullpage}

\title{The Quantum Linear Chain}
\author{Anthony Mezzacappa}
\date{September 3, 2020}

\begin{document}

\maketitle

\noindent Following the usual quantization procedure
\begin{align*}
    q_n(t) &\longrightarrow \hat{q_n}(t) \\
    p_n(t) &\longrightarrow \hat{p_n}(t)
\end{align*}
where $\hat{q_n}(t)$ and $\hat{p_n}(t)$ are now operators satisfying the commutation relation:
\begin{align*}
    [ \hat{q_n} (t), \hat{p_{n'}}(t)] &= i\hbar \delta_{n,n'} \\
    [ \hat{q}_n (t), \hat{q}_{n'} (t) ] &= 0 \\
    [ \hat{p}_n (t), \hat{p}_{n'} (t) ] &= 0 
\end{align*}

\noindent -- i.e., the Poisson brackets have been replaced by commutators.

\noindent Focusing on $\hat{q_n} (t)$:
\begin{equation*}
    \hat{q_n} (t) = \sum_l ( \hat{a}_l (t) u_{ n, l } + \hat{a}^\dagger_l (t) u^*_{n, l} )
\end{equation*}
where $\hat{a}^\dagger_l$ are now operators whose meaning is to be determined.

\noindent Similarly,
\begin{equation*}
    \hat{p}_n (t) = m \Dot{\hat{q}} (t) = i m \sum_l \omega_l \left( \hat{a}_l (t) u_{ n, l } - \hat{a}^\dagger_l u^*_{ n, l }  \right)
\end{equation*}

% Page 1a

\noindent Let's invert the relationships
\begin{align}
    \hat{q}_n (t) &= \sum_l \left( \hat{a}_l (t) u_{ n, l } + \hat{a}^\dagger_l u^*_{ n, l } \right) \label{eq1} \\ % Equation 1
    \hat{p}_n (t) &= i m \sum_l \omega_l \left( \hat{a}_l (t) u_{ n, l } - \hat{a}^\dagger_l u^*_{ n, l } \right) \label{eq2} % Equation 2
\end{align}

\noindent Multiply \eqref{eq1} and \eqref{eq2} by $ u^*_{ n, l' } $ and sum over $n$
\begin{align}
    \sum_{ n = 1 }^N u^*_{ n, l' } \hat{q}_n (t) &= \sum_l \biggl\lbrace \hat{a}_l (t) {\underbrace{ \sum_{ n = 1 }^N u^*_{ n, l' } u_{ n, l } }_{\delta_{ l l' } }} + \hat{a}^\dagger_l (t) {\underbrace{ \sum_{ n = 1 }^N u^*_{ n, l' } u^*_{ n, l } }_{ \delta_{ l, -l' } }} \biggr\rbrace \nonumber \\
    &= \hat{a}_{l'} (t) + \hat{a}^\dagger_{-l'} (t) \label{eq3} % Equation 3
\end{align}

\begin{align}
    \sum_{ n = 1 }^N u^*_{ n, l' } \hat{p}_n (t) &= i m \sum_l \omega_l \biggl\lbrace \hat{a}_l (t) \sum_{ n = 1 }^N u^*_{ n, l' } u_{ n, l } - \hat{a}^\dagger_l (t) \sum_{ n = 1 }^N u^*_{ n, l' } u^*_{ n, l } \biggr\rbrace \nonumber \\
    &= i m \omega_l \left[ \hat{a}_{l'} (t) - \hat{a}^\dagger_{-l'} (t)  \right] \label{eq4} % Equation 4
\end{align}

% Page 1b
Then from \eqref{eq3} and \eqref{eq4} 
\begin{equation}
    \hat{a}_{l'} (t) + \hat{a}^\dagger_{-l'} (t) = \sum_{n = 1}^N u^*_{n, l'} \hat{q}_n(t) \label{eq5}
\end{equation} 
\begin{equation}
    \hat{a}_{l'} (t) - \hat{a}^\dagger_{-l'} (t) = \sum_{n = 1}^N \frac{1}{im\omega} u^*_{n, l'} \hat{p}_n(t) \label{eq6}
\end{equation} 

Adding \eqref{eq5} and \eqref{eq6},
\begin{equation}
    \hat{a}_{l'} (t) = \frac{1}{2} \sum_{n = 1}^N u^*_{n, l'} \left [ \hat{q}_n(t) + \frac{1}{im\omega}  \hat{p}_n(t) \right ]
\end{equation}

Since
\begin{equation*}
    \hat{q}_n (t) = [\hat{q}_n (t)]^\dagger
\end{equation*}
\begin{equation*}
        \hat{p}_n (t) = [\hat{p}_n (t)]^\dagger
\end{equation*}

We have,
\begin{equation*}
    \hat{a}^\dagger_{-l'} (t) = \frac{1}{2} \sum_{n = 1}^N u_{n, l'} \left [ \hat{q}_n(t) - \frac{1}{im\omega}  \hat{p}_n(t) \right ]
\end{equation*}

% Page 2
The relationship we learned between $a_l (t)$ and $(q_n (t), p_n (t))$ now becomes the operator relation: 
\begin{equation*}
    \hat{a}_l (t) = \frac{1}{2} \sum_{n = 1}^N u^*_{n, l} \left [ \hat{q}_n(t) + \frac{1}{im\omega}  \hat{p}_n(t) \right ]
\end{equation*}

We can use this to determine the commutator associated with the operator $\hat{a}(t)$:
\begin{eqnarray*}
       \left [ \hat{a}_l (t), \hat{a}^\dagger _{l'}(t) \right ] &=&   \frac{1}{4} \sum_n \sum_{n'} u^* _{n,l} u_{n',l'} \left \{  \left[ \hat{q}_n(t) + \frac{i}{m\omega_l}\hat{p}_n (t) \right] \left[ \hat{q}_{n'}(t) - \frac{i}{m\omega_{l'}}\hat{p}_{n'}(t) \right] \right. \\
       && \left. - \left[ \hat{q}_{n'}(t) - \frac{i}{m\omega_{l}}\hat{p}_{n'}(t) \right]\left[ \hat{q}_n(t) + \frac{i}{m\omega_l}\hat{p}_n (t) \right] \right \} \\
       &=& \frac{1}{4} \sum_n \sum_{n'}u^* _{n,l} u_{n',l'} \left \{ \hat{q}_n (t)\hat{q}_{n'} (t) - \hat{q}_{n'} (t) \hat{q}_n (t)  - \frac{i}{m\omega_l} \left( \hat{q}_n (t)\hat{p}_{n'} (t) - \hat{p}_{n'}(t) \hat{q}_n (t) \right) \right. \\
       && \left. + \frac{i}{m\omega_l} \left( \hat{p}_n (t)\hat{q}_{n'} (t) - \hat{q}_{n'}(t) \hat{p}_n (t) \right)  -\left(\frac{i}{m\omega_l}\right)^2 \left( \hat{p}_n (t)\hat{p}_{n'} (t) - \hat{p}_{n'}(t) \hat{p}_n (t) \right) \right \} \\
       &=& \frac{1}{4} \sum_n \sum_{n'}u^* _{n,l} u_{n',l'} \left \{ -\frac{i}{m \omega_l} i \delta_{nn'} + \frac{i}{m\omega_l} (-i \delta_{nn'})\right \} \\ 
       &=& \frac{1}{2} \frac{1}{m \omega_l} \sum_{n}u^* _{n,l} u_{n',l'} \\ 
       &=& \frac{1}{2m\omega_l} \delta_{ll'}
\end{eqnarray*}

\begin{equation*}
    \left[ \sqrt{2m\omega_l} \hat{a}_l (t), \sqrt{2m\omega_{l'}} \hat{a}^\dagger _{l'}(t) \right ] = \delta{ll'}
\end{equation*}

Define, 
\begin{equation*}
    \hat{b}_l (t) \equiv \sqrt{2m\omega_l} \hat{a}_l (t)
\end{equation*}

\newpage
Then,
\begin{eqnarray*}
     \left[ \hat{b}_l (t), \hat{b}^\dagger _{l'}(t) \right ] &=& \delta_{ll'} \\
      \hat{b}(t) \hat{b}^\dagger(t) -  \hat{b}^\dagger (t) \hat{b}(t) &=& 1  
\end{eqnarray*}

Let's now guess the Hamiltonian in terms of $\hat{b}_l (t)$:
\begin{eqnarray*}
    H &=& \sum_l m \omega _l ^2 \left [ \hat{a}_l ^\dagger (t) \hat{a}_l (t) + \hat{a}_l (t) \hat{a}_l ^\dagger (t)  \right ] \\
    &=& \sum_l m \omega _l ^2 \frac{1}{2m\omega_l} \left [ \hat{b}_l ^\dagger (t) \hat{b}_l (t) + \hat{b}_l (t) \hat{b}_l ^\dagger (t)  \right ] \\ 
    &=& \frac{1}{2} \sum_l \omega_l \left [ 2\hat{b}_l ^\dagger (t) \hat{b}_l (t) + 1 \right] \\
    &=& \sum_l \omega_l  \left [ \hat{b}_l ^\dagger (t) \hat{b}_l (t) + \frac{1}{2} \right] 
\end{eqnarray*}

Using the creation and annihilation operator for each mode $\hat{b}_l ^\dagger$ and $\hat{b}_l$, we can create the multiparticle state
\begin{equation*}
    \ket{ n_1, n_2, n_3, \dots } = \prod_l \ket{n_l} 
\end{equation*}
where, 
\begin{eqnarray*}
    \ket{n_l} &=& \frac{1}{\sqrt{n!}} \left( \hat{b} _l ^\dagger \right)^{n_l} \ket{0_l}  \\
    \braket{ n_l | n_l } &=& 1
\end{eqnarray*}
The state $\ket{0_l}$ is the vacuum state for the mode $l$. 
\end{document}